%==================================================
% Standard
%==================================================
\documentclass[11pt]{article} % Page setup
\usepackage[top=1in, bottom=1in, left=1in, right=1in]{geometry} % Margin setup
\usepackage[square,sort,comma,numbers]{natbib}  % Reference Formatting
\usepackage{fancyhdr} % Header setup
\usepackage[toc,page]{appendix}
\usepackage[hang,flushmargin]{footmisc}
\usepackage{textpos}

%==================================================
%Typesetting Packages
%==================================================
\usepackage[colorlinks=true,linkcolor=Blue,citecolor=Red,urlcolor=Blue]{hyperref} % sets the color of hyperlinks
\usepackage{url} %Allows urls to be displayed properly
\usepackage{amsmath,amssymb} % Highly recommended as an adjunct to serious mathematical typesetting; amssymb provides extended symbol collection
\usepackage{textcomp} % Package supports Text Companion fonts (baht, bullet, copyright, musicalnote, onequarter, section, and yen)
\usepackage{color} % Provides foreground (text, rules, etc.) and background color management
\usepackage{colordvi} % Allows for using named colors when typsetting output
\usepackage[usenames,dvipsnames,svgnames,table]{xcolor} % Extends the color package; several color tints, shades, tones, and mixes of arbitrary colors
\definecolor{arsenic}{rgb}{0.23, 0.27, 0.29}
\definecolor{color18}{rgb}{0.5,0.5,0.5} % Defines a color named "color18"
\usepackage{bm} % bold math
\usepackage{siunitx} % si units package 
\newcommand{\SIper}{\SI[per-mode=symbol]} % Special command for the SI unit package
\usepackage{enumerate} % Allows for the enumerate style to be changed 
%\usepackage{enumitem}
\usepackage{alphalph}
\usepackage{setspace}
\usepackage{blindtext}
\usepackage{soul}
\usepackage[tocindentauto]{tocstyle}
\usepackage{algorithm}
\usepackage{algpseudocode}
\usepackage{multirow}
\usepackage{mathtools}
\usepackage{minted}
\usepackage{afterpage}
\usepackage{float}
\usepackage{changepage}
\numberwithin{equation}{section}
%\usepackage{mathptmx}

% ==================================================
%Figure packages
% ==================================================
\usepackage{graphicx} % Include figure files
\usepackage{rotating} % Performs all rotations including complete figures and tables with their captions ON ONE PAGE
\usepackage{lscape} % Rotates the page contents but not the page number; can be applied across many pages
\usepackage[font=scriptsize,format=plain,labelfont=bf]{caption} % Caption customization (also for tables)
\usepackage{subcaption} % Provides support for subcaptions
\usepackage{float} % Improved interface for defining float objects (figures and tables)
\usepackage{wrapfig} % Allows figures or tables to have text wrapped around them
\setlength{\parskip}{5pt} %Space between paragraphs
\setlength{\parindent}{0pt} %Paragraph Indent size
\usepackage{fancybox}
\usepackage{mdframed}
\usepackage{makeidx}
\usepackage{tablefootnote}

% ==================================================
% Table packages
% ==================================================
\usepackage{longtable} % allow multipage tables
\usepackage{multicol}
\usepackage{multirow} % create tabular cells spanning multiple rows
\usepackage{dcolumn} % Align table columns on decimal point
\usepackage{threeparttable}
\usepackage{lipsum,booktabs}

% ==================================================
% Section Formatting 
% ==================================================
\usepackage{sectsty}
\allsectionsfont{\bfseries} %Sets ALL section font

%Title, author, date formats	
	\newcommand{\mytitle}[1]{\title{\bf{\textsf{#1}}}}
	\newcommand{\myauthor}[1]{\author{\textsf{#1}}}
	\newcommand{\mydate}[1]{\textsf{\date{#1}}}
	\newcommand{\mytoday}{\textsf{\today}}

%Section header formats	(Numbered)
	\newcommand{\mysection}[2]{\vspace{-5px}\section{#1}\label{#2}\vspace{-10px}}
	\newcommand{\myssection}[2]{\vspace{-5px}\subsection{#1}\label{#2}\vspace{-5px}}
	\newcommand{\mysssection}[2]{\vspace{-5px}\subsubsection{#1}\label{#2}\vspace{-5px}}

%Section header formats	(Unnumbered)	
	\newcommand{\mysectionN}[2]{\vspace{-5px}\section*{#1}\label{#2}\vspace{-10px}}
	\newcommand{\mysectionnonum}[2]{\vspace{-5px}\section*{#1}\label{#2}\vspace{-10px}}
	\newcommand{\myssectionN}[2]{\vspace{-5px}\subsection*{#1}\label{#2}\vspace{-5px}}
	\newcommand{\myssectionnonum}[2]{\vspace{-5px}\subsection*{#1}\label{#2}\vspace{-5px}}
	\newcommand{\mysssectionN}[2]{\vspace{-5px}\subsubsection*{#1}\label{#2}\vspace{-5px}}
	\newcommand{\mysssectionnonum}[2]{\vspace{-5px}\subsubsection*{#1}\label{#2}\vspace{-5px}}

%Paragraph header formats		
	\newcommand{\mypart}[2]{\part{#1}}
	\newcommand{\mypara}[2]{\paragraph{#1}}
	\newcommand{\myparaN}[2]{\paragraph*{#1}}
	\newcommand{\myparanonum}[2]{\paragraph*{#1}}

%References, etc.	
	\newcommand{\myref}[2]{\hyperref[#2]{#1~\ref{#2}}}
	\newcommand{\myrefexp}[3]{\hyperref[#3]{{#1}~{#2}}}
	\newcommand{\hilight}[1]{\colorbox{yellow}{#1}}
	\newcommand{\myfilename}[1]{\texttt{\textsf{#1}}}
	\newcommand{\mywebsite}[1]{\myfilename{\bf #1}}
	\newcommand{\commandline}[1]{\texttt{> #1}}
	\newcommand{\plusplus}[1]{#1{}\texttt{++}}

%Callout Box Creator	
	\newcommand{\callout}[3]
	{
	\begin{wrapfigure}{#1}{#2\textwidth}
	\vspace{-20pt}
	\centering
	\fbox{\parbox{#2\textwidth}{#3}}
	\vspace{-10pt}
	\end{wrapfigure}
	}

% ==================================================
% Track changes and commenting packages/commands
% ==================================================
\usepackage{todonotes}
	\newcommand{\sn}[1]{\todo[color=magenta!40,caption={}]{#1}}
	
\usepackage{comment}


	%Other colors available 
	%cyan
	%green
	%white
	%darkgray
	%brown
	%olive
	%pink
	%purple
	%violet

